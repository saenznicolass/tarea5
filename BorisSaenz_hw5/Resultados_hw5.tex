\documentclass[notitlepage,letterpaper,12pt]{article}
\usepackage[spanish]{babel} 

%guia para informes de lb de ondas 

\usepackage[utf8]{inputenc} 
\usepackage[T1]{fontenc} 
\usepackage[normalem]{ulem}
\usepackage[spanish]{babel}
\useunder{\uline}{\ul}{}
\providecommand{\e}[1]{\ensuremath{\times 10^{#1}}}
\usepackage{textcomp}
\usepackage{gensymb}
\usepackage[colorlinks=true,urlcolor=blue,linkcolor=blue]{hyperref}
\usepackage{url} 
\usepackage{amsmath}
\usepackage{amsfonts}
\usepackage{amssymb}
\usepackage{physics} 
\usepackage{graphicx}
\usepackage{epstopdf}
\usepackage{multirow}
\usepackage[export]{adjustbox}
\usepackage{geometry}     
\geometry{letterpaper}     
\usepackage{fancyhdr} 
\pagestyle{fancy}
\chead{\bfseries {}}
\lhead{}
\lfoot{\it Tarea 5 Métodos Computacionales.}
\cfoot{ }
\rfoot{Universidad de los Andes}

\voffset = -0.25in
\textheight = 8.0in
\textwidth = 6.5in
\oddsidemargin = 0.in
\headheight = 20pt
\headwidth = 6.5in
\renewcommand{\headrulewidth}{0.5pt}
\renewcommand{\footrulewidth}{0,5pt}

\begin{document}
\title{Tarea 5 Métodos Computacionales}
\author{
\textbf{Boris Nicolas Saenz Rodriguez\thanks{e-mail: \texttt{bn.saenz10@uniandes.edu.co}}}\\
\date{25 de mayo de 2017}
\textit{Universidad de los Andes, Bogotá, Colombia}\\
} 

\maketitle 


\newpage
\section{Gráficas}

\subsection{punto 1}

A continuacion se presenta la grafica obtenida con el archivo  1, en la cual se evidencian los valores de X, Y y R para el circulo mas grande posible en el plano con las moleculas.


\begin{figure}[h!]
  \centering
   \includegraphics[scale= 0.8]{fig1.png}
  \caption{Circulo para archivo 1 }
  \label{ini}
\end{figure}

\newpage

Seguido se evidencia el histograma en 2D correspondiente al archivo 1.

\begin{figure}[h!]
  \centering
   \includegraphics[scale= 0.8]{hist1.png}
  \caption{Histograma en 2D para archivo 1}
  \label{c1t100c}
\end{figure}
\newpage


A continuacion se presenta la grafica obtenida con el archivo  2, en la cual se evidencian los valores de X, Y y R para el circulo mas grande posible en el plano con las moleculas.

\begin{figure}[h!]
  \centering
   \includegraphics[scale= 0.8]{fig2.png}
  \caption{Caso 2, condicion fija, t = 100.}
  \label{fig: cobre}
\end{figure}
\newpage


Ahora, el histograma en 2d que corresponde al archivo 2 de la tarea.
\begin{figure}[h!]
  \centering
   \includegraphics[scale= 0.8]{hist2.png}
  \caption{Histograma en 2D para archivo 2}
  \label{c1t100p}
\end{figure}
\newpage

\subsection{punto 2}

A continuacion se presenta la grafica con los datos y el modelo de  los mejores parametros de R y C obtenidos para el punto 2 de la tarea.
\begin{figure}[h!]
  \centering
   \includegraphics[scale= 0.8]{pru.png}
  \caption{Modelo mejores parametros R y C}
  \label{fig: cobre}
\end{figure}

\newpage


Asimismo se muestra la grafica de recorridos para resistencias y capacitancias junto con la grafica de recorrido para variables reales

\begin{figure}[h!]
  \centering
   \includegraphics[scale= 0.8]{RrRc.png}
  \caption{Recorridos para resistencias y capacitancias}
  \label{fig: cobre}
\end{figure}

\begin{figure}[h!]
  \centering
   \includegraphics[scale= 0.8]{Rqt.png}
  \caption{Recorridos para las variables reales}
  \label{fig: cobre}
\end{figure}

\newpage

Tambien se muestran los histogramas de capacitancia y resistencia

\begin{figure}[h!]
  \centering
   \includegraphics[scale= 0.8]{hC.png}
  \caption{Histograma de capacitancia}
  \label{fig: cobre}
\end{figure}

\begin{figure}[h!]
  \centering
   \includegraphics[scale= 0.8]{hR.png}
  \caption{Histograma de resistencia}
  \label{fig: cobre}
\end{figure}

\newpage 

Por ultimo se presentan las graficas de verosimilitud para resistencia y capacitancia

\begin{figure}[h!]
  \centering
   \includegraphics[scale= 0.8]{Vr.png}
  \caption{Modelo mejores parametros R y C}
  \label{fig: cobre}
\end{figure}


\begin{figure}[h!]
  \centering
   \includegraphics[scale= 0.8]{Vc.png}
  \caption{Modelo mejores parametros R y C}
  \label{fig: cobre}
\end{figure}


\newpage


%\section{Referencias}


%\bibliographystyle{unsrt} % estilo de las referencias 
%\bibliography{mybib.bib} %archivo con los datos de los artículos citados


%\bibliography{mybib.bib} %archivo con los datos de los artículos citados

% Forma Manual de hacer las referencias
% Se escribe todo a mano...
% Descomentar y jugar

%\begin{thebibliography}{99}
%\bibitem{Narasimhan1993}Narasimhan, M.N.L., (1993), \textit{Principles of
%Continuum Mechanics}, (John Willey, New York) p. 510.

%\bibitem{Demianski1985}Demia\'{n}ski M., (1985), \textit{Relativistic
%Astrophysics,} in International Series in Natural Philosophy, Vol 110, Edited
%by \textit{D. Ter Haar}, (Pergamon Press, Oxford).
%\end{thebibliography}


%Fin del documento
\end{document}

